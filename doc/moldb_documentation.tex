\documentclass[12pt]{article}

\usepackage[top=2cm,bottom=2cm,left=2cm,right=2cm]{geometry}

\usepackage{verbatim}
\usepackage{graphicx}

\title{MolDB - Database for Molecular Calculations}
\author{Volodymyr P. Sergiievskyi}

\begin{document}

\maketitle

%Molecular Database strucutre

\setcounter{section}{-1} 

\section{Motivation}

Typically, each calculation has several stages, e.g:
\begin{enumerate}
\item extract pdbs form the sources
\item assign the force-field
\item calculate QM charges
\item start 3DRISM calculations
\item calculate solvation free energy and RDFs
\end{enumerate}

However, relatively often, one wants to try calculations with several different parameters  at some stages, which will lead to variety of different results at the next stages.
%Also, most of calculations are connected to one or another molecule.
However, often before starting the calculations, one have no idea which exactly parameters are necessary to use.
One can come to conclusion only after series of tests had been performed. And the process is continious and endless: the more one investigates the problem, the more different parameters of the algorithms one whant to try. 
% But you would like to distinguish them. 
%Also, you would like to distinguish the results, based on different calculation variants.
Here is an example:

3DRISM calculations need the following stages:
\begin{itemize}
\item extract coordinates from the Pdb files
\item assign a Force Field
\item calculate Charges
\item run 3DRISM
\end{itemize}

But, there are more than one force-field (CHARM, OPLS, etc).
Also, there are more than one way to assign the charges (CHELPG, Muliken etc).
As well, one could want to perform  3DRISM calculations with different parameters (Accuracy, Buffer, etc).

So, as a result, to distinguish different methods, one would need to create folders like


\verb|3DRISM_CHARM_CHELPG_Accuracy1e-4_Buffer10A|

Suh a long names do not seem to be good readable. 
Moreove, you have already a potential source of errors. 
For example, typically it is very simple to forget the exact order in which you list the parameters. 
And the resulting folder names are often hardly distinguishible, which is a potential source of errors
(Compare 


\verb|3DRISM_CHARM_CHELPG_Accuracy1e-4_Buffer10A|

and 


\verb|3DRISM_CHELPG_CHARM_Accuracy1e-4_Buffer10A|

)

Another problem with this folder format can appear if one wants to extend the parameters set (which will normally happen to almost any calculation).
In that case, one need to use longer names. 
E.g let's imaginem that at some point one  think that it could be perfect to perform the geomertry optimization before the calculations.
In that case, the folders will have one additional parameter: geometry optimization on or off, e.g.

\verb|3DRISM_opt_CHARM_CHELPG_Accuracy1e-4_Buffer10A|

At some point, it may turn out, that the method you use by default was not the best. 
Thus, you should try also another one.
In that case, you will need to add something

\verb|3DRISM_opt631p_CHARM_CHELPG_Accuracy1e-4_Buffer10A|

Except of obvious disadvantages (long, hardly readable, hardly distinguisable names), there also appear a problem with compatibility.
So, if in previous calculations optimization was not performed, or it was performed with ``default'' parameters, the explicit name (opt631p) should not be included in the name of folder.
But according to the ``full'' naming convention, it should be included explicitly.

So, one need either to rename all the folders according to the ``full'' convention, or remember, that some short names in reality correspond to some default parameters (e.g. ``opt'' means ``opt631p'', but not anything else).
This also reduce readability and comprehansibility of the folder name, and can lead to the errors.


Although such naming conventions can be successfully used in some cases, for small projects with limited number of parameter variations, it is obvious, that in general it would be preferable to have more consistent and universal way to manipulate the calculations.



%Btw, all your previous calculations should be renamed, because it is not clear now, which optimization was used in them....
%Also, in some systems there are limitations for the maximal length of the filename... 

\section{Hashable data storage}

%I propose the following structure:
Instead of listing all the parameters in the file (or folder) name, it could be better to store them in a separate file.
Still, it is reasonable to have one folder per method (because typically one need to store somewhere input/output and supplementary files for each method).
However, there are no reason to include all the parameters into the folder name. 


Each molecule's files are located in a separate folder.
Indide this folder there are subfolders with methods.
Inside the methods' subfolders there is a file, which describes the calculation chain.
The name of the folder, except of the method name, contains the hash, which uiquely determines the calculation chain and all calculation parameters.

So, the following structure is proposed:

\begin{verbatim}
-mol1
      -method1_HASH1
             -description.txt
	- file1
	- file2
	-output1
	-output2
	
       -method1_HASH2
             -description.txt
              … 	 

        -method2_HASH3
             -description.txt
              … 
       … 

-mol2
-mol3
... 
\end{verbatim}

To retrieve the data from the database one need a special script (python?), which will go through all the records (subset of records), read description.txt files and build the table like this:

\begin{verbatim}
molecule method1_param1   method1_param2 ... method1_paramN   method2_param1 methodN_paramN
name1              val11                   val22                   …..                                                             valN
name1              val 21                   val22                 …...                                                             val2N
…. 
name2    ….
…..
\end{verbatim}

From such a table it is easy to select the records with given parameters (one can write special python (or even shell) function for this.)

To put data to the database one also should use the script, which will calculate the HASH and insert the data into appropriate folder.... 

\section{Sources, Methods and Prototypes}

In each investigation one would probably have the following types of data:
\begin{itemize}
\item  
\textbf{Sources.}
 Files, taken from external sources, which are used as an input for the calculations
 
 \item
 \textbf{Methods.}
The data structure, described in the previous section, where each method with each set of parameters is located in a separate folder.

\item
\textbf{Prototypes.}
The folder, which contain the scripts to peform the calculations and to process the calculation data.

\end{itemize}

\subsection{Sources}

In principle, Sources folder can be organized in an arbitrary way.
Actually, it depends much on a nature of the source data.
In is suggested to include the references to the parpers to the data where possible. 

\subsection{Prototypes}

The prototypes section contain the command sequences to process the data. 
The main type of such sequences in the \emph{Chain} files, which contain the description of the sequence of calculations.

%%%%%%%%%%%%%%%%%%%%%%%%%%%%%%%%%%%%%%%%%%%%%%%%%%%%%%%%%%
%%%%%%%%%%%%%%%%%%%%%%%%%%%%%%%%%%%%%%%%%%%%%%%%%%%%%%%%%%

\section{Free Energy Calculations with 3DRISM}

%%%%%%%%%%%%%%%%%%%%%%%%%%%%%%%%%%%%%%%%%%%%%%%%%%%%%%%%%%
%%%%%%%%%%%%%%%%%%%%%%%%%%%%%%%%%%%%%%%%%%%%%%%%%%%%%%%%%%

We will describe the details of the database organization on example of the solvation free energies calculation with 3DRISM.

For the analyis we choose the set of 504 molecules from the paper 
\emph{Mobley et al. J. Chem. Theory Comput. 2009, 5, 350-358}.
The structures of the molecules and the solvation free energies are provided in the supporting information to the paper.
The data is saved in the appropriate structure in the \emph{Source} folder.

To perform the calculations you will need:
\begin{itemize}
 \item MolDB
 \item python
 \item RISM-MOL-3D ( download from \verb#http://compchemmpi.wikispaces.com/RISM-MOL-3D#
 \item octave (for analyzing the data )
\end{itemize}

\subsection{Chain file}

The first step is to convert the data to the appropriate formats and save in to the database (forler structure).
To do it, in the MolDB are used so-called \emph{Chain} files.
Each chain file describes the cain of calculations (actually - workflow).
The chain file contains the description of the parameters, the relations between the methods and the scripts which are used for the calculation.
Let's look at the \emph{Source.chain} file, available in the \emph{Prototypes} folder:

\begin{verbatim}
PROJECT CanadaMol

BEGIN CHAIN PROTOTYPE
1 METHOD
RUN METHOD 1


BEGIN METHOD Source
ID 1

	BLOCK DEPENDENCIES
	0 RECORDS	
	ENDBLOCK DEPENDENCIES

	BLOCK PARAMETERS
	2 RECORDS
		ref = LIST $MOLDB_PROJECTS/CanadaMol/Sources/JCTC_2009_5_350/JCTC_5_350.bib
		molecule = FILE molecules.lst
	ENDBLOCK PARAMETERS

	BLOCK SCRIPTS
	3 RECORDS

       SCRIPT copyFiles
	         ....
   	   ENDSCRIPT	

   	  SCRIPT parseMol2
   	     ...
  	  ENDSCRIPT

	  SCRIPT parsePrmtop
	     ...
	  ENDSCRIPT


	ENDBLOCK SCRIPTS

END METHOD ParseSource

END CHAIN PROTOTYPE
\end{verbatim}


The first line contain the project name (must be the same as the folder name of the project).
Next lines contain the descriptuin of the \emph{Chain}, which starts with the phrase \emph{BEGIN CHAIN PROTOTYPE}.
The chain consists of the methods, which can depend on each other (e.g. results of the calculations of one method can be used as an input for other methods).
In the next line it is written, how many methods there are in the chain (\emph{1 METHOD})
One of the methods is a \emph{run method}, e.g. the method which will be runned by the scripts.
All other methods are used just to provide dependencies/extra parameters.
In our case there is only one method, 1 is its identifier ( \emph{RUN METHOD 1} )

After that we see the definition of the methods.
In our case we have only one method.
Each method has both: 1) Name. 2) identifier (ID)
The name is used to create the propper folder structure.
The identifier is used to describe the dependencies of the methods.
The name of the method is given in the first line of the method definition 
(\emph{BEGIN METHOD Source}), the identifier is given in the next line (\emph{ID 1})

The method definition contains three main \emph{BLOCKS}.
Each block starts with the sentence \emph{BLOCK blockname}, 
where \emph{blockname} is one of \emph{DEPENDENCIES}, \emph{PARAMETERS} or \emph{SCRIPTS}.
The next line of the block defenition describes the number of records in the block, e.g \emph{5 RECORDS}.
After that the records are defined.

\subsubsection{BLOCK DEPENDENCIES}
 
In this block the dependencies of the methods are described.
Each line is one record.
In our example we have no dependencies.
In general case the syntax is the following:
\begin{verbatim}
  METHOD methodID AS nickname
\end{verbatim}
The parameters of the dependencies are inherited and the folder of the dependencies is available in the scripts under the name \emph{nickname}
For example, if some method is dependent on our \emph{Source} method we have:
\begin{verbatim}
BLOCK DEPENDENCIES
1 RECORD
  METHOD 1 AS SourceNickName
ENDBLOCK DEPENDENCIES  
\end{verbatim}

\subsubsection{BLOCK PARAMETERS}

In the parameters block the parameters of the method are described.
Each parameter can have several values.
The program will iterate over the values and the scripts will be runned for all possible combinations of the values.

Each parameter is given at the separate line.
The definition is in the form
\emph{name = values }.
The parameter name \textbf{must be lowercase}.

The values can be either given explicitly or read from file.
In the first case one need to use the keyword \emph{LIST} and then space separated values.
In the second case the keyword \emph{FILE} is used and then the name of file is given.
The values in the file are one-per-line.

Example:
\begin{verbatim}
BLOCK PARAMETERS
3 RECORDS
   molecule = FILE molecules.lst
   closue = LIST KH HNC
   charges = LIST yes no
ENBLOCK PARAMETERS
\end{verbatim}


\textbf{Also important:}
Some values of the parameters can have \emph{default} values.
In that case these values are not used for HASH generation.
This is important, if you already performed some calculations, and then want to extend them using one more degree of freedom (one more parameter).
But your calculations were performed with some value of that parameter (which you did not include into the chain file).
In that case you can add the value of this parameter to the list followed by the \emph{(default)} keyword.

Example: 
Let's extend the previous example to the case of different buffer/spacing.
Let in the previous calculations we used Buffer=15 and Spacing = 0.5 Angstroem, and now want to try different buffers and different spacings (be we dont want re-do the calculations which are already done).

The parameters block for that case:
\begin{verbatim}
BLOCK PARAMETERS
5 RECORDS
   molecule = FILE molecules.lst
   closue = LIST KH HNC
   charges = LIST yes no
   buffer = LIST 10 15 (default)  20
   spacing = LIST 0.2 0.3 0.5 (default)
ENBLOCK PARAMETERS
\end{verbatim}
 
 The values buffer=15 and spacing =0.5 will not be included to the HASH, and thus the same folders as before will be used.

\subsubsection{BLOCK SCRIPTS}

In the scripts block  the scripts which perform the calculations are defined.
Each script starts with the keyword \emph{SCRIPT scriptname}, and ends with \emph{ENDSCRIPT}
The scripts are usuall bash scripts.
For each script the separate file will be created.
They will be runned in the method's folders.
All the parameters of the method are available for the scripts as local variables.
Also, the folder names of dependencies are available as local variables ( the name of these variables are the names which are given in \emph{DEPENDENCIES} section).
Also, some special variables are available, namely:
\emph{\$FOLDER} - the name of the current folder
\emph{\$PROTOTYPES} - the name of the \emph{Prototypes} folder (where chain files are located).

Also, some global variables related to the MolDB are available (after the installation):
\emph{\$MOLDB\_ROOT},\emph{\$MOLDB\_BIN}, \emph{\$MOLDB\_PROJECTS}

The example of the scripts:
\begin{verbatim}
		SCRIPT copyFiles
			
			echo $molecule

			Source=$MOLDB_PROJECTS/CanadaMol/Sources/JCTC_2009_5_350

			cp $Source/charged_mol2files/$molecule.mol2 .
			cp $Source/prmcrd/$molecule.prmtop .
			cp $Source/prmcrd/$molecule.crd .

		ENDSCRIPT	
\end{verbatim}



\begin{verbatim}
		SCRIPT parseMol2
		# parses Mol2 files and gets coors and charges 

		cat $molecule.mol2 | grep ATOM -A 1000 | grep BOND -B 1000 | head -n-1 | tail -n+2 >mol2.tmp
		cat mol2.tmp | gawk '{print $2}' > atomnames_full.txt
		cat atomnames_full.txt | tr -d [0-9] > atomnames.txt
		cat mol2.tmp | gawk '{print $3" "$4" "$5}' > $molecule.coors
		cat mol2.tmp | gawk '{print $9}' > $molecule.charges
		
		ENDSCRIPT
\end{verbatim}


\begin{verbatim}

		SCRIPT parsePrmtop
			#Uses : makePdb, ambertop2rism.py
			# They are located in ~/460GB/Development/utils folder (should be in Paths)
			# Note: utils folder should be distributed with the database

			echo $FOLDER

			makePdb $molecule.coors atomnames.txt > $molecule.pdb

			P=$MOLDB_PROJECTS/CanadaMol/Prototypes

			python $P/ambertop2rismmol.py $molecule.prmtop $molecule.pdb $molecule.rism

			cat $molecule.rism | gawk '{print $6}' > charges.txt
			cat $molecule.rism | gawk '{print $4}' > sigma.txt
			cat $molecule.rism | gawk '{print $5}' > epsilon.txt
		ENDSCRIPT			
\end{verbatim}
Other examples one may find in the chain files provided with the MolDB.

\subsection{Running the calculations}

The calculations are performed in three steps: 
\begin{enumerate}
\item Creating the folder structure
\item Creating the scripts wich perform all the calculations
\item Running the scripts
\end{enumerate}

\subsubsection{moldb\_createFolders}

So, the first step is to create folder structure in the \emph{Methods} folder.
To do this, in the \emph{Prototypes} folder type:
\begin{verbatim}
   moldb_createFolders Source.chain
\end{verbatim} 

The script will create the folders and copy all the scripts to them.
If you want create folders not for all molecules, but for some subset, you can give coma-separated options after the chain file, e.g.
\begin{verbatim}
  moldb_createFolders Source.chain 'molecule=111_trichloroethane'
\end{verbatim}

You can use also keywords \emph{LIST} and \emph{FILE}, e.g.:
\begin{verbatim}
  moldb_createFolders Source.chain 'molecule=LIST 111_trichloroethane 1112_tetrachloroethane'
\end{verbatim}
or 
\begin{verbatim}
  moldb_createFolders Source.chain 'molecule=FILE short.lst'
\end{verbatim}

In our example there is only one parameters (molecule).
In general you can give any number of coma-separated options, e.g:

\begin{verbatim}
  moldb_createFolders Source.chain 'molecule=FILE short.lst,closure=KH'
\end{verbatim}

After the folders are created, you would probably want to check that everything is OK with them, before continue the calculations.

To do that you can copy one of the folder names and cd to that folder...

\subsubsection{moldb\_getRef}

Alternatively, you can use \emph{moldb\_getRef} command, to get the folder for a molecule/molecules with specific attributes, e.g
\begin{verbatim}
  moldb_getRef Source.chain 'molecule=111_trichloroethane'
\end{verbatim}
The syntax of the coma separated list is the same as above.

\subsubsection{PARAMETERS,DEPENDENCIES,chain.imp}


If you cd to the folder you can see the list of the files.
It include all the script files (which were provided in the \emph{Source.chain}, and additionally - three specific files:
\begin{itemize}

\item PARAMETERS - this file contains all the parameters of the simuation which will be available to the script:
\begin{verbatim}
PROJECT=CanadaMol
FOLDER=$MOLDB_PROJECTS/CanadaMol/Methods/111_trichloroethane/Source/107913302
PROTOTYPES=$MOLDB_PROJECTS/CanadaMol/Prototypes
molecule=111_trichloroethane
ref=$MOLDB_PROJECTS/CanadaMol/Sources/JCTC_2009_5_350/JCTC_5_350.bib
\end{verbatim}

\item DEPENDENCIES - this file contain all the references to the folder which were given in the DEPENDENCIES section.

\item chain.imp - The file, analogous to the Source.chain, BUT, instead of lists of variables there are some concreete values, e.g:
\begin{verbatim}
PROJECT CanadaMol
BEGIN CHAIN IMPLEMENTATION
1 METHODS
RUN METHOD 107913302

BEGIN METHOD Source
	ID 107913302
	BLOCK DEPENDENCIES
	0 RECORDS
	ENDBLOCK DEPENDENCIES
	BLOCK PARAMETERS
	2 RECORDS
		molecule = 111_trichloroethane
		ref = $MOLDB_PROJECTS/CanadaMol/Sources/JCTC_2009_5_350/JCTC_5_350.bib
	ENDBLOCK PARAMETERS
	BLOCK SCRIPTS
	3 RECORDS
		.... 
	ENDBLOCK SCRIPTS
END METHOD Source

END CHAIN IMPLEMENTATION

\end{verbatim}

\end{itemize}


Also, you would probably want to run your scripts, to check that there are no errors in them.
Just do it like ./copyFiles or ./prepareMol2 while you are in the molecule's folder.

To return from the molecule's folder you can 

cat PARAMETERS

copy the PROTOTYPES line and cd there, e.g:
\begin{verbatim}
cd $MOLDB_PROJECTS/CanadaMol/Prototypes
\end{verbatim}



\subsubsection{moldb\_runScript}

After creation the folders you can run the scripts.
We have three scripts: copyFiles, parseMol2, parsePrmtop

To run we script one can use \emph{moldb\_runScript} command:

\verb# moldb_runScript Source.chain copyFiles  #

However, this command does not run the script, it only creates the script file which will go through all the folders and run appropriate script in them.
The script name is 
  \emph{ProjectName\_ChainName\_scriptName.sh}

  In our case: 
   \emph{CanadaMol\_Source\_copyFiles.sh}

If you want to run the script only for specific molecules, you can use the command with options, e.g.

\verb# moldb_runScript Source.chain copyFiles  'molecule=FILE short.lst'#

The scipt like this will be generated:

\begin{verbatim}
#!/bin/bash
#
#  XPARAM: molecule=FILE short.lst
#
FOLDERS="
$MOLDB_PROJECTS/CanadaMol/Methods/1112_tetrachloroethane/Source/158372845
$MOLDB_PROJECTS/CanadaMol/Methods/111_trichloroethane/Source/107913302
$MOLDB_PROJECTS/CanadaMol/Methods/111_trifluoro_222_trimethoxyethane/Source/652622319
"

N=$(echo $FOLDERS | wc -w )
count=0
P=$(pwd)
for FOLDER in $FOLDERS
do
	echo Molecule $count of $N "(" $((count*100/N)) percent done ")"
	cd $FOLDER
	. ./PARAMETERS
	. ./DEPENDENCIES
	./copyFiles
	count=$((count+1))
done
\end{verbatim}

You can run in by just typing:

\verb$  ./CanadaMol_Source_copyFiles.sh $

Also, probably you would like to use more that one thread.
In that case use 

\verb# moldb_runScript Source.chain copyFiles -p 2  #

Where 2 is the number of processes.
Then several script files will be created named like

\begin{verbatim}
./CanadaMol_Source_copyFiles_p0.sh
./CanadaMol_Source_copyFiles_p1.sh
\end{verbatim}
 
and the file \verb#./CanadaMol_Source_copyFiles.sh# which run both of them in parallel.

\textbf{Hint:} Don't use parallelization for file copying: it makes no sence because several processes will try acces the HDD simultaniously and efficiency will be low.
However, for calculations (3DRISM or so) this option is very useful if you have more than one CPU core (AND enough memory, don't forget!) 

So, finally to do everything we need with Source.chain we have to run the following commands:

\begin{verbatim}
moldb_createFolders Source.chain

moldb_runScript Source.chain copyFiles
./CanadaMol_Source_copyFiles.sh

moldb_runScript Source.chain parseMol2
./CanadaMol_Source_parseMol2.sh

moldb_runScript Source.chain parsePrmtop
./CanadaMol_Source_parsePrmtop.sh

\end{verbatim}

Also, additionally:
We need to parse the table with experimental/MD values and also store them to the Source folders.
This can be done by the separate script which uses \emph{moldb\_getRef}.

The script name is \emph{storeTable1}:
\begin{verbatim}
#!/bin/bash

Table1=../Sources/JCTC_2009_5_350/SI/Table1/Table1.txt

IFS=$'\n'
Cols="molecule"$IFS
Cols=$Cols"DeltaG_elec"$IFS"ErrDeltaG_elec"$IFS
Cols=$Cols"DeltaG_vdw"$IFS"ErrDeltaG_vdw"$IFS
Cols=$Cols"DeltaG_hydr"$IFS"ErrDeltaG_hydr"$IFS
Cols=$Cols"DeltaG_expt"

IFS=$'\n'
for line in $(cat $Table1)
do
	
	molecule=$(echo $line | gawk '{print $1}' )

	folder=$(moldb_getRef Source.chain  molecule=$molecule)

	N=0
	for col in $Cols
	do
		N=$((N+1))
		val=$(echo $line | gawk "{print \$$N}" | tr − "  -")

		echo 'echo '$val ' > '$folder'/'$col'.dat'
	done

done
\end{verbatim}

This script does not do any actions, it only print the commands to do them (just to give you an opportunity that everything is OK before running).

So, the commands to run are:

\begin{verbatim}
./storeTable1 > storeTable1_run
chmod 777 storeTable1_run
./storeTable1_run
\end{verbatim}

\subsection{3DRISM.chain}

Now, we have all the Source data prepared, we can run the 3DRISM calculations.
We run the 3DRISM for each molecule for such variations: 
with/without charges, with KH/HNC closure (4 variants in total).

The 3DRISM.chain file is the following:
\begin{verbatim}
PROJECT CanadaMol

BEGIN CHAIN PROTOTYPE
2 METHODS
RUN METHOD 2

FROM Source.chain IMPORT METHOD 1 AS 1

BEGIN METHOD 3DRISM
ID 2

	BLOCK DEPENDENCIES
	1 RECORD
		METHOD 1 AS Source
	ENDBLOCK DEPENDENCIES

	BLOCK PARAMETERS
	2 RECORDS
		closure = LIST HNC KH
		charges = LIST yes (default) no
	ENDBLOCK PARAMETERS

	BLOCK SCRIPTS
	3 RECORDS

		SCRIPT prepareInput
		...
		ENDSCRIPT 


		SCRIPT run3DRISM
		....
		ENDSCRIPT

		SCRIPT clearCalculations
			.....
		ENDSCRIPT

	ENDBLOCK SCRIPTS

END METHOD 3DRISM

END CHAIN PROTOTYPE

\end{verbatim}

We see, that in this file are definded two methods: Source and 3DRISM, and the method 3DRISM depends on Source (see BLOCK DEPENDENCIES)

The method SOurce is \emph{imported} from the file Source.chain:
\verb# FROM Source.chain IMPORT METHOD 1 AS 1 #
 
The syntax of IMPORT command:

\verb# FROM file IMPORT METHOD methodIDinFile AS newMethodID #


There are three scripts in the 3DRISM.chain file:

\begin{itemize}

\item prepareInput - copies files which are necesary for tehe calculation to the method's folder

\item run3DRISM - perform the 3DRISM calculations

\item clearCalculations - remove input/output files (necessary for reducing the size of the database)

\end{itemize}

\subsubsection{prepareInput}

\begin{verbatim}
		SCRIPT prepareInput

			echo $FOLDER

			cp $PROTOTYPES/3DRISM/* .
			cat parameters.tmpl | sed "s:SED_CLOSURE:$closure:g" > parameters.txt

			cp $Source/$molecule.rism mol.rism
			
			if [ $charges == no ]; then

				cat mol.rism | gawk '{print $1" "$2" "$3" "$4" "$5}' > mol.tmp
				N=$(cat mol.rism | wc -l)

				rm -f zero.txt			

				for((i=0;i<$N;i++))
				do
					echo 0 >> zero.txt
				done
				
				multicol mol.tmp zero.txt > mol.rism

			fi


		ENDSCRIPT 
\end{verbatim}

As we see, the script copies the rism input file from the corresponding Source folder (given by \emph{\$Source} variable)

If there are no charges it also generates the file with the last (charge) columns with zeros.

Also, the script changes closure in the parameters file to the correct value.

\subsubsection{run3DRISM}


\begin{verbatim}
		SCRIPT run3DRISM

			echo $FOLDER
			echo $molecule $closure

			tic=$(date +%s)

			
			multiGridMain parameters.txt 'SoluteStructureFile="mol.rism";' >multiGrid.out &
			pid=$!
			TimeLimit=30  #% *10 = 300 sec = 5min
			

			for((i=0;i<=$TimeLimit;i++))
			do
				sleep 10
				N=$(ps -A | grep $pid | wc -l)
				
				if [ $N == 0 ]; then
					break;
				fi
	
				echo -n .
			done

			echo $i $pid

			
			if [ $i -ge $TimeLimit ]; then
				echo ITERATION DIVERGED 
				kill -9 $pid
			fi

			calculateFreeEnergySimple mol_in_water_ > freeEnergy.out

			toc=$(date +%s)

			cat freeEnergy.out | grep HNC | gawk '{print $2}' > HNC.dat
			cat freeEnergy.out | grep KH | gawk '{print $2}' > KH.dat
			cat freeEnergy.out | grep GF | gawk '{print $2}' > GF.dat
			cat freeEnergy.out | grep VUA | gawk '{print $2}' > PMV.dat

			tail -n4 freeEnergy.out
			echo Time Elapsed=$((toc-tic))

			rm -f *.3d

		ENDSCRIPT
\end{verbatim}


The script performs the 3DRISM and solvation free energy calculations and stores the results to the separate files (HNC.dat, KH.dat, GF.dat, PMV.dat).

Also it removes the 3d files - 3D distributions of oxygen/hydrogen (to save the disk space - files can be $>10 M$, and you should multiply this values by the number of molecules (504).

\subsubsection{clearCalculations}

\begin{verbatim}
		SCRIPT clearCalculations

			# clears all intermediate files used for the calculation
			# prepareInput will be necessary before next start of run3DRISM

			rm -f water.slv
			rm -f *.grd
			rm -f waterRDFs.txt
			rm -f parameters.txt
			rm -f mol.rism
			rm -f *.out

		ENDSCRIPT
\end{verbatim}


Removes the input/output files (except the results), saves the disc space.
Is necessary, if you want to copy your database somewhere (saves space).

\subsubsection{run the method}

To run the method use the following scripts:
\begin{verbatim}

moldb_createFolders 3DRISM.chain

moldb_runScript 3DRISM.chain prepareInput
./CanadaMol_3DRISM_prepareInput.sh 

moldb_runScript 3DRISM.chain run3DRISM -p 2
./CanadaMol_3DRISM_run3DRISM.sh 

\end{verbatim}

Here -p 2 means that the calculations will be parllelized to 2 cores. 
Use another value for another hardware... 

\section{Visualizing and analyzing Results}

\subsection{moldb\_printResults}

To view results of the calculations you can use \emph{moldb\_printResults} command. 

The syntax is:
\verb# moldb_printResults chain rows columns file [ restrictions  ]  #

The reusults are printed in the table view.
Rows and cols are the coma-separated lists of parameters.

The command will read the \emph{file} from all the folders described by \emph{chain}.

Example:
1)

\verb$ moldb_printResults 3DRISM.chain molecule charges,closure HNC.dat  $

Output: 

\begin{verbatim}

molecule                                charge/HNC charge/KH nocharge/HNC nocharge/KH 
1112_tetrachloroethane                    16.148     20.162    18.267       22.088    
111_trichloroethane                       14.898      18.44    16.585       19.939    
111_trifluoro_222_trimethoxyethane        17.153      22.68    23.571       28.518    
111_trifluoropropan_2_ol                  6.5384     10.677    16.742       19.967    
111_trimethoxyethane                      12.788     18.214    21.428       26.1
...
\end{verbatim}

The script uses the sed commands from the \emph{aliases.txt} to make the headers of the table look better.


2)

\verb$ moldb_printResults 3DRISM.chain molecule closure HNC.dat  'charges=yes' $

Output:
\begin{verbatim}
molecule                                HNC       KH        
1112_tetrachloroethane                    16.148    20.162  
111_trichloroethane                       14.898     18.44  
111_trifluoro_222_trimethoxyethane        17.153     22.68  
111_trifluoropropan_2_ol                  6.5384    10.677  
111_trimethoxyethane                      12.788    18.214  
1122_tetrachloroethane                    14.884    19.005  
112_trichloro_122_trifluoroethane         18.765    22.537  
112_trichloroethane                       12.128     15.65  
11_diacetoxyethane                        11.421    18.174  
11_dichloroethane                          12.06    15.201  
....

\end{verbatim}

If you want to view the results, which has less then 1 parameters (e.g. Source ), you can use the following syntax with \emph{dummy} column, e.g:

\verb# moldb_printResults Source.chain molecule dummy DeltaG_vdw.dat 'dummy=MD_nocharge'  #

Output:

\begin{verbatim}
molecule                                MD_nocharge 
1112_tetrachloroethane                      1.54    
111_trichloroethane                         1.88    
111_trifluoro_222_trimethoxyethane          2.31    
111_trifluoropropan_2_ol                    2.49    
111_trimethoxyethane                        1.66    
1122_tetrachloroethane                      1.47    
112_trichloro_122_trifluoroethane           1.85    
112_trichloroethane                         1.51    
11_diacetoxyethane                          1.62    
11_dichloroethane                           1.81    
....
\end{verbatim}

\subsection{Visualization using octave (matlab)}

To visualize results you can use octave.
Below is the example (all the commands - in octave):

-1st  step: add the path to your MolDB bin folder to your ~/.octaverc file, e.g

\begin{verbatim}
echo "path(path,'"$MOLDB_BIN"');" >> ~/.octaverc
echo "path(path,'"$MOLDB_ROOT/utils"');" >> ~/.octaverc
\end{verbatim}

-2nd step - run octave

-3rd step - build the references to the folders from Source.chain and 3DRISM.chain:

\begin{verbatim}
 Ref3DRISM=moldb_getRef('3DRISM.chain');
 RefSource=moldb_getRef('Source.chain');
\end{verbatim}

-4th step - filter the results
To select some subset from the references use the \emph{moldb\_filter} command:

\verb$ I = moldb_filter(Ref3DRISM,{'closure','HNC','charges','yes'}); $
The command returns the indeces of the specified records in the Ref3DRISM. 
These indeces can be used for all data related to the Ref3DRISM (e.g. data files loaded for this data)


The parameters to filter are given as a cell array in a key-value pairs \verb#{key1, val1, key2, val2 , ...}#

-5th step
 load the data files to the cell arrays:
 \begin{verbatim}
   Exp_cell = moldb_load(RefSource,'DeltaG_expt.dat');
   HNC_cell = moldb_load(Ref3DRISM,'HNC.dat');
   PMV_cell = moldb_load(Ref3DRISM,'PMV.dat');
 \end{verbatim}
 
 Note: the data is returned in the cell array.
 To convert this arrays to the matrices use \verb# cell_get_values#:
 \begin{verbatim}
   Exp = cell_get_values(Exp_cell);
   HNC = cell_get_values(HNC_cell);
   PMV = cell_get_values(PMV_cell);
 \end{verbatim}
 
-6th step: 
  plot the results
  
 \begin{verbatim}
   plot(Exp,HNC(I),'s',Exp,Exp,'k')
   xlabel('\Delta G_{expt}')
   ylabel('\Delta G_{HNC}')
  
 \end{verbatim}
   
 \includegraphics[width=0.6\textwidth]{examp1}

-7th step: 
fit the results

\begin{verbatim}
J=isfinite(HNC(I));
a=myregress(HNC(I(J))-Exp(J),PMV(I(J)))
plot(PMV(I),HNC(I)-Exp,'s',PMV(I),a*PMV(I),'k')
xlabel('PMV')
ylabel('\Delta G_{HNC} - \Delta G_{expt}')
S=nanstd(HNC(I)-a*PMV(I) - Exp)

text(50,30,[ 'y = ' num2str(a) '*PMV    std = ' num2str(S) 'kcal/mol' ])
 
\end{verbatim}

 \includegraphics[width=0.6\textwidth]{examp2}

\end{document}